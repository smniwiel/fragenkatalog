\section{Altklausurfragen}
\subsection{WS 12/13 | 20 Punkte}
\begin{enumerate}
    \item Wie lautet das zweite Newtonsche Gesetz?
    \item Zeigen Sie, dass aus der Drehimpulserhaltung beim Keplerproblem folgt, dass die Bahnkurven der Planeten in einer Ebene liegen.
    \item Was können Sie über den zu einer zyklischen Koordinate $q_k$ gehörigen generalisierten Impuls $p_k$ aussagen?
    \item Wieviel generalisierte Koordinaten brauchen Sie, um die Lagrangefunktion eines mechanisches Systems mit $n$ Freiheitsgraden aufzustellen?
    \item Geben Sie die Hamiltonfunktion eines eindimensionalen harmonischen Oszillators an. Schreiben Sie bitte, was die von Ihnen verwendeten Größen bedeuten.
    \item Geben Sie zwei Beispiele für nichtintegrable Systeme - also für Systeme bei denen chaotische Bahnen auftreten können.
    \item Was besagt das Gesetz von Hagen-Poiseullle?
    \item Wann nennt man eine Strömung stationär?
    \item Wann nennt man eine Flüssigkeit inkompressibel?
    \item Was besagt die Maxwellgleichung div $\vec{B}$ = 0 anschaulich?    
\end{enumerate}
\subsection{WS 13/14 | 20 Punkte}
\begin{enumerate}
    \item Wie lautet das zweite Newtonsche Gesetz?
    \item Ein Massepunkt der Masse $m$ bewegt sich kräftefrei im dreidimensionalen Raum. Geben Sie die Newtonsche Bewegungsgleichung an und lösen Sie diese.
    \item Was versteht man in der Mechanik unter dem Satz  "actio=reaction" ? 
    \item Auf welchen Bahnen bewegen sich Massen im Kepler-Problem? Skizzieren Sie dazu das effektive Kepler-Potential?
    \item Berechnen Sie das elektrische Feld eines unendlich langen Drahtes mit homogener Ladungsdichte $\lambda$.
    \item Was besagt die Maxwellgleichung div $\vec{B}$ = 0 anschaulich?
    \item Was versteht man unter transversalen Wellen?
    \item Wann nennt man eine Flüssigkeit inkompressibel?
    \item Geben Sie ein Beispiel für ein Problem in der Mechanik, in dem chaotisches Verhalten auftreten kann.
    \item Skizzieren Sie den Poincare-Schnitt von zwei ungekoppelten eindimensionalen harmonischen Oszillatoren.
    \item Gegeben sei die Lagrangefunktion
    \begin{equation*}
        L (q_1, q_2, q_3, \dot{q_1}, \dot{q_2}, \dot{q_3}) = \frac{m}{2} \dot{q_1}^2 +  \frac{m}{2} \dot{q_2}^2 \frac{m}{2} \dot{q_3}^2 - q_1 q_2 - q_2^2 \medspace .
    \end{equation*}
            Welche der drei Koordinaten $q_1$, $q_2$, $q_3$ sind zyklisch?\\
            Was können Sie über den zu einer zyklischen Koordinaten $q_k$ gehörigen generalisierten Impuls $p_k$ aussagen?
    \item Wie lautet die Hamiltonfunktion für einen eindimensionalen harmonischen Oszillator?
    \item Vergleichen Sie die aus dem Lagrang-Formalismus hergeleiteten Bewegungsgleichungen mit dem aus dem Hamilton-Formalismus hergeleiteten Bewegungsgleichungen ( Wie unterscheiden sich Anzahl und Ordnungen der Differentialgleichung?).
    \item Berechnen Sie das Integral
    \begin{equation*}
        \int_{-\infty}^{\infty} dx cos(x - \frac{\pi}{2}) \updelta (x - \frac{\pi}{2}) \medspace . 
    \end{equation*}
    \item Wie lautet das Coulombsche Gesetz?
\end{enumerate}


