\section{Altklausurfragen}
\subsection{WS 12/13 | 20 Punkte}
\begin{enumerate}
    \item Wie lautet das zweite Newtonsche Gesetz?
    \item Zeigen Sie, dass aus der Drehimpulserhaltung beim Keplerproblem folgt, dass die Bahnkurven der Planeten in einer Ebene liegen.
    \item Was können Sie über den zu einer zyklischen Koordinate $q_k$ gehörigen generalisierten Impuls $p_k$ aussagen?
    \item Wieviel generalisierte Koordinaten brauchen Sie, um die Lagrangefunktion eines mechanisches Systems mit $n$ Freiheitsgraden aufzustellen?
    \item Geben Sie die Hamiltonfunktion eines eindimensionalen harmonischen Oszillators an. Schreiben Sie bitte, was die von Ihnen verwendeten Größen bedeuten.
    \item Geben Sie zwei Beispiele für nichtintegrable Systeme - also für Systeme bei denen chaotische Bahnen auftreten können.
    \item Was besagt das Gesetz von Hagen-Poiseullle?
    \item Wann nennt man eine Strömung stationär?
    \item Wann nennt man eine Flüssigkeit inkompressibel?
    \item Was besagt die Maxwellgleichung div $\vec{B}$ = 0 anschaulich?
    
\end{enumerate}
